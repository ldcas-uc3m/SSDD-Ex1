\section{Compilación}
Para compilar los archivos hacemos uso de un make file. La idea es poder cambiar el sistema del servidor y el cliente si es necesario pero mantener la independencia del cliente de estos cambios. Por eso, el archivo de claves, se compila en una librería .so.

El código del makefile está incluido en la entrega y corresponde a un fichero estandar de compilación pero con algunos añadidos para la creación de esta librería dinámica.

Lo primero es la inclusión de la flag "-fPIC" que si es soportado por la maquina, genera un codigo independiente de la posición que permite el linkeado dinámico y evitando cualquier posible límite en el tamaño de la tabla global de offset.

Despues es la inclusión de la flag "-lclaves" para linkear la librería dinámica de libclaves con el archivo principal.

Por último, está la compilación de claves.o en libclaves.so que se ha realizado con el siguiente código (al que se le ha añadido la flag "-shared"):

\begin{lstlisting}
    libclaves.so: claves.o
        $(CC) -shared $(CFLAGS) $(LDFLAGS) $(LDLIBS) $^ -o $@
\end{lstlisting}

Cabe destacar, que el fichero de make incluye también un make clean para eliminar todos los archivos de compilación intermedia.